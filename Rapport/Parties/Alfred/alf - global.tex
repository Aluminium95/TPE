Le programme, un mélange d'algorithmie, de données et de pourcentages scientifiques, et la réalisation du design.

Le programme est une simulation d'une souche de levure qui se divise occupant de plus en plus d'espace au fur et à mesure que le temps passe. Elle est ensuite exposée aux ultraviolets. Ce qui nous donne la mutation de certaines cellules résistant à ce nouvel antécédent. Les autres, n'ayant pas les moyens de subsister meurent.

Pour se faire, le scientifique a donc rechercher des sources sur les cellules de levure. Ces recherches montrèrent ainsi la fiabilité du programme dans le domaine scientifique. Le temps de division et de vie d'une cellule de levure, le taux de mutation et de mort quand il y a exposition aux ultraviolets en furent les plus importantes.

Le programmeur a pu ainsi inscrire ces données. Mais le design restait à faire.  Le styliste s'est donc mis à la tâche. Son travail a consisté à représenter la cellule, le quadrillage et les boutons de menu de la façon dont il pensait du mieux qu'ils soient. Il fallait aussi demander comment représenter la division, la mutation et la mort des cellules.
