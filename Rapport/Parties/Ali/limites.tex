Le programme tel qu'il existe actuellement est fini. Mais fini ne veut pas dire qu'il est terminé.
Un programme n'est jamais terminé, il y a toujours des choses à apporter, des bugs à corriger, même les versions finales ne le sont pas. Chaque nouvelle version apporte son lot ne nouveautés, et son lot de frustration du fait de ne pas avoir pu finir tout ce que l'on voulait créer.

Le programme aujourd'hui n'est plus modifié jusqu'à la présentation orale, de manière à ne pas avoir des fonctions qui ne sont pas soigneusement décrites dans ce rapport.

Mais nous avons quand même dressé une liste des limites qu'il a actuellement, et des fonctions que l'on pourrait lui ajouter. Cette liste a été crée au fur et à mesure de la conception par toute l'équipe, chacun apportant sa vision de la simulation. Mais le plus gros des idées, viennent des personnes à qui nous avons fait tester le programme, généralement des amis ou nos parents, qui chacun dans leur domaines, ont critiqué.

\begin{description}
  \item[La forme des cellules] : les cellules sont uniquement représentées comme des rectangles de couleur. Il eu été intéressant de pouvoir mettre une image plus réaliste en conservant le jeu des couleurs. Il faudrait pouvoir définir ses propres images de cellule, ce qui permettrait de renforcer la généricité du programme.
  \item[Les caractéristiques non rendues] : le programme ne peut pas montrer de façon visible toutes les caractéristiques des cellules. Par exemple la taille de la membrane, le caryotype ou certains métabolismes. Du fait de la « surcharge » d'information que cela engendrerait. Il aurait fallu pouvoir « zoomer » sur une cellule en particulier pour avoir des informations précises sur celle-ci.
  \item[La portabilité] : le manque de portabilité du programme rend son installation sur un ordinateur plus difficile que prévue. C'est un problème non négligeable pour la distribution, il nous a par exemple été impossible de vous livrer le programme sous la forme d'un exécutable unique, du fait des librairies utilisées, qui auraient du être installées séparément, ce qui aurait été beaucoup moins simple qu'une vidéo de démonstration.
  \item[La gestion du temps] : il manque une gestion automatisée du temps qui soit plus simple et intuitive. Actuellement la configuration du programme demande une petite gymnastique mentale qui est désagréable, surtout quand on sait qu'il est possible de se l'éviter. Une autre fonctionnalité de gestion du temps serait de pouvoir modifier la vitesse de la simulation durant l'exécution du programme, pour ralentir durant l'exposition aux UV, et accélérer durant des phases de division par exemple.
  \item[Les divisions] : les divisions «~poussantes~» n'existent que pour les 4 côtés, et non pas les diagonales, il en résulte que les colonies ont une forme de losange.
  \item[La 3D] : en réalité les cellules se divisent dans l'espace tri-dimensionnel. Les levures par exemple forment des colonies sous forme de «~bulles~», qui ne dépassent pas une certaine hauteur du fait de la gravité, mais qui peut quand même être conséquente. Plutôt que de faire de la 3D, on aurait pu imaginer une légende de densité avec des couleurs, mais elle pose le problème des autres légendes qui utilisent déjà des couleurs.
  \item[L'environnement] : modifier les UV c'est bien, mais ce n'est pas vraiment suffisant, dans le principe, le logiciel devrait gérer la nutritivité de l'environnement, la température, la capacité d'ajouter des substances comme les antibiotiques. En résumer, pouvoir agir beaucoup plus sur l'environnement de la simulation.
  \item[Random] : souvent en biologie quand il faut ensemencer des cultures, on utilise des billes de verre et on secoue, ce qui a pour effet de disperser la solution de manière aléatoire et donc statistiquement quasi-uniforme. Il faudrait utiliser une fonction de ce type pour l’ensemencement de départ de la simulation.
\end{description}
