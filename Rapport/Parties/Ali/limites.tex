Le programme tel qu'il existe actuellement est fini. Mais fini ne veut pas dire qu'il est terminé.
Un programme n'est jamais fini, il y a toujours des choses à apporter, des bugs à corriger, même les versions finales ne le sont pas. Chaque nouvelle version apporte son lot ne nouveautés, et son lot de frustration du fait de ne pas avoir pu finir tout ce que l'on voulait créer.

Le programme aujourd'hui n'est plus modifié jusqu'à la présentation orale, de manière à ne pas avoir des fonctions qui ne sont pas soigneusement décrites dans ce rapport.

Mais nous avons quand même dressé une liste des limites qu'il a actuellement, et des fonctions que l'on pourrait lui ajouter. Cette liste a été crée au fur et à mesure de la conception par toute l'équipe, chacun apportant sa vision de la simulation. Mais le plus gros des idées, viennent des personnes à qui nous avons fait tester le programme, généralement des amis ou nos parents, qui chacuns dans leur domaines, ont critiqué.

\begin{quotation}
	« bla bla bla »
	\begin{flushright}
		Sébastien Lopez
	\end{flushright}
\end{quotation}

Nous avons donc imaginé ceci …

\begin{quotation}
	« Un jeu vidéo qui vous laisse rêveur devant tant de réalisme »
	\begin{flushright}
		J.V.N.
	\end{flushright}
\end{quotation}


Nous avons donc imaginé cela …
