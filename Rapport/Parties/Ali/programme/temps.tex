L'ordinateur ne gère pas le temps nativement. Et notre système de boucle principale non plus.
Nous savons juste qu'entre chaque cycle d'affichage il s'écoule environ 20 millisecondes, idem pour les cycles d'exécution, mais avec un «~environ~» encore plus large.
  
Il a donc fallu faire des conversions. Du fait que le programme soit configurable, il a été d'autant plus difficile de le faire, car au lieu d'ajuster les paramètres, il a fallu trouver toutes les opérations pour que les calculs fonctionnent dans tous les cas.

Par exemple quand on demande que toutes les $N$ secondes (réellement écoulées) une cellule se divise, il faut :
  \begin{itemize}
    \item $N \times 50$ pour donner des secondes (20 millisecondes entre chaque tour)
    \item À chaque tour la cellule vérifie que son nombre de tour de boucle accumulé est divisible par 
le nombre de tour de boucle nécessaire à une division : $50N$.
  \end{itemize}
  
Ceci ce complexifie quand on demande en plus que 1 seconde réelle donne $N$ minutes virtuelles.
Car tous les calculs doivent prendre en charge ce nouveau paramètre.

Il n'a d'ailleurs pas été possible de le gérer partout faute de temps, c'est pour cela que l'on gère la division des cellules en secondes réelles dans la configuration.


