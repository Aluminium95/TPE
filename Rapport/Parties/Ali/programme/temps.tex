L'ordinateur ne gère pas le temps nativement. Heureusement la SDL apporte certaines fonctions 
pour attendre un certain nombre de millisecondes. La gestion du temps 
dans une simulation est primordiale, et complexe.

Notre approche a été de raisonner en cycles cellulaires, ce qui est plus simple, mais moins fluide.

Plutôt que de mettre des «~wait~» SDL dans notre boucle principale, nous avons choisi de lancer 
deux processus en parallèle de la gestion des évènements grâce à la fonction SDL «~timer~» : 
  \begin{itemize}
    \item L'affichage, lancé toutes les 20 millisecondes
    \item L’exécution de simulation, lancé toutes les 20 millisecondes aussi
  \end{itemize}
  
Bien sûr il a fallu faire des conversions en heure, en minutes, en secondes. Du fait que le programme soit configurable, il a été d'autant plus difficile de le faire, car au lieu d'ajuster une bonne foi pour toutes les paramètres, il a fallu trouver toutes les opérations pour que les calculs fonctionnent tout le temps.

Par exemple quand on demande que toutes les $N$ secondes (réelles) une cellule se divise, il faut :
  \begin{itemize}
    \item $N \times 50$ pour donner des secondes (20 millisecondes entre chaque tour)
    \item À chaque tour la cellule vérifie que son nombre de tour de boucle accumulé est divisible par 
le nombre de tour de boucle nécessaire à une division : $50N$.
  \end{itemize}
