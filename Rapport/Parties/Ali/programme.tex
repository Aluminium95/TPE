\section{Présentation des outils}
  Pour réaliser notre programme, il faut des outils logiciels.
  Voici une liste exhaustive des outils dont nous avons besoin, et le 
  nom ainsi que la description de ceux que nous avons choisi : 
    \begin{description}
      \item[Un ordinateur] : celui d'Aliaume 
      \item[Un langage de programmation] : nous avons choisi le Vala. Ce langage est en réalité un
        un méta-langage, un traducteur le transforme en un véritable langage de programmation : le C.
        Le C a des avantages, comme la rapidité, et des inconvénients, comme l'insécurité. Le Vala est une 
        surcouche qui permet de simplifier et de clarifier le code, en utilisant des paradigmes de programmation
        que le C n'a pas à travers une bibliothèque nommée GObject, écrite en C.
      \item[Une bibliothèque pour afficher des pixels à l'écran] : nous avons choisi la SDL, qui est une bibliothèque pour le C qui permet d'afficher des pixels à l'écran. Elle est plutôt basique, et par la même très simple. On peut donc faire tout ce que l'on veut … si l'on y met le temps de réfléchir pixel par pixels.
    \end{description}
  
  Ces choix ont été fait pour des raisons de performances, mais aussi parce que le C est un langage qui est éprouvé, et dont les vices sont tous connus. Idem pour la SDL. Tous deux sont très bien documentés. Mais, aussi parce que nous avions déjà utilisé ces outils auparavant.

\section{Programmer le contexte}
  \input{Parties/Ali/programme/contexte.tex}

\section{Simulation et boucle principale}
  \input{Parties/Ali/programme/boucle.tex}
\section{Rectangle, la base}
  Nous savons que la majorité des éléments de la simulation seront affichés à l'écran.
Or pour afficher quelque chose sur une surface, il faut un certain nombre d'informations : 
\begin{itemize}
  \item Position $(x,y)$ sur l'écran
  \item Taille $(w,h)$
\end{itemize}

Nous dessinerons uniquement des rectangles, pour des raisons de simplicité, et parce que nous n'avons pas besoin d'autres formes géométriques dans notre simulation\footnote{En effet, même si on définit les cellules comme des cercles, pour afficher une image, il faut le rectangle de la taille de l'image, même si l'image elle même est un rond sur fond transparent}.

Avec ces variables, on peut ajouter plusieurs méthodes qui seront utiles :
\begin{itemize}
  \item déplacer($x$,$y$); $\rightarrow$ redéfinit la position $(x,y)$
  \item move ($x$,$y$); $\rightarrow$ effectue une translation par le vecteur $(x,y)$
\end{itemize}

Nous avons donc notre première classe : «~Rectangle~», qui sera la base pour afficher des objets à l'écran.

  
\section{Le menu}
  
L'exemple du menu étant simple et parlant, nous montrerons comment nous avons pensé 
ce module. Dans un menu, il y a des boutons. Donc il faut un objet Menu\footnote{De manière à être 
utilisable dans d'autres programmes, ou pouvoir faire plusieurs menus}, et un objet Bouton. Un Bouton est un objet affiché à l'écran, il faut donc le faire hériter de Rectangle, un objet Menu a une position sur l'écran, il faut donc aussi qu'il hérite de Rectangle.

De cette manière on a maintenant des classes comme ceci :

  
\subsection{Les boutons}
	Nous allons créer la classe \texttt{Bouton} qui hérite de \texttt{Rectangle}, mais qui apporte son lot de nouveautés, à savoir tout ce qui est relatif à un bouton : 
	\begin{itemize}
		\item une image pour le bouton normal
		\item une image pour le bouton activé
		\item une variable pour savoir si le bouton est activé
		\item une variable de type ennumération\footnote{Voir annexe \ref{DefEnum} sur les énnumération, qui parle du menu en exemple} pour savoir l'action qu'il entraine
	\end{itemize}
	
	\texttt{Bouton} lui-même n'a pas vraiment de fonctions, tout le code se trouvera dans le gestionnaire des boutons, \texttt{Menu}. Mais avant de parler du menu, il faut définir les actions, or nous ne savons pas encore lequelles il y aura, nous allons donc créer une énnumération, qui sera traitée dans la boucle des évènements. Ajouter une action demandera d'ajouter un item à l'énumération, et un traitement de cet item dans la boucle évènementielle.
	
\subsection{Le menu}
	Le menu gère les boutons, son code n'est pas aussi complexe qu'on pourrait se l'imaginer, il contient un tableau de boutons, une fonction \texttt{affiche} qui affiche le menu et une fonction \texttt{recupBouton} qui retourne le bouton qui se situe sous le curseur\footnote{En effet, on doit gérer nous même ce sur quoi clique le curseur}.
	
	La seule difficulté réside dans le fait de dessiner correctement les boutons, et au bon endroit. Pour cela il faut savoir que l'écran de la SDL se comporte comme un repère orthonormé, à ceci près qu'il a pour origine le coin haut gauche de la fenêtre, et que son axe des Y est inversé par rapport aux graphiques habituels : 
	
	Après ça, il faut aussi savoir que le menu a une position, donc quand on place un \texttt{Bouton}, il faut le placer en fonction de la position du \texttt{Menu}, et en fonction de la taille d'un \texttt{Bouton}. La méthode suivante est utilisée : quand on crée un bouton, on lui donne deux images, une \texttt{Action}, et la position du bouton vaut celle de la simulation plus $N$ fois la largeur d'un bouton en $x$, en fonction du numéro du bouton.
	
	\begin{figure}[H]
	\centering
	\includegraphics[width=26em]{Images/menu.png}
	\caption{Capture d'écran du menu}
\end{figure}
	
	C'est bon, nous avons un menu fonctionnel, reste à connecter tout cela avec la boucle évènementielle et c'est terminé. Nous ne parlerons pas de la boucle évènementielle, car elle est en réalité simple. Globalement elle demande si le curseur a cliqué, si oui où. Après elle regarde sur quelle partie du programme le curseur est positionné, et un système de conditions décide de ce qu'il faut faire. Ce code n'est pas intéressant, simplement parce que c'est une suite de conditions imbriquées sans grande complexité.

    
\section{La gestion des évènements}
  \input{Parties/Ali/programme/event.tex}
\section{La gestion du temps}
  L'ordinateur ne gère pas le temps nativement. Heureusement la SDL apporte certaines fonctions 
pour attendre un certain nombre de millisecondes. La gestion du temps 
dans une simulation est primordiale, et complexe.

Notre approche a été de raisonner en cycles cellulaires, ce qui est plus simple, mais moins fluide.

Plutôt que de mettre des «~wait~» SDL dans notre boucle principale, nous avons choisi de lancer 
deux processus en parallèle de la gestion des évènements grâce à la fonction SDL «~timer~» : 
  \begin{itemize}
    \item L'affichage, lancé toutes les 20 millisecondes
    \item L’exécution de simulation, lancé toutes les 20 millisecondes aussi
  \end{itemize}
  
Bien sûr il a fallu faire des conversions en heure, en minutes, en secondes. Du fait que le programme soit configurable, il a été d'autant plus difficile de le faire, car au lieu d'ajuster une bonne foi pour toutes les paramètres, il a fallu trouver toutes les opérations pour que les calculs fonctionnent tout le temps.

Par exemple quand on demande que toutes les $N$ secondes (réelles) une cellule se divise, il faut :
  \begin{itemize}
    \item $N \times 50$ pour donner des secondes (20 millisecondes entre chaque tour)
    \item À chaque tour la cellule vérifie que son nombre de tour de boucle accumulé est divisible par 
le nombre de tour de boucle nécessaire à une division : $50N$.
  \end{itemize}

    
    
    
     
