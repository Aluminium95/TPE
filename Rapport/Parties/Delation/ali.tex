Au départ je voulais faire mon TPE avec Yann, mais je n'avais encore aucune idée de sujet à traiter. Nous avons par la suite eu l'obligation de former des groupes de trois, et nous avons donc intégré Alfred au groupe, sans qu'aucun de nous n'ai d'idée, même vague de ce que l'on pouvait faire.

Notre plus gros problème lors de ce TPE fût de trouver un sujet qui nous intéressent tous, qui soit à la portée de tous, et en général, d'en proposer un tout court. Durant les premiers mois, j'ai vraiment eu la sensation que le groupe tournait à vide. Durant quelques séances, l'objectif était uniquement de définir une problématique. 

  Finalement nous nous sommes orientés vers la programmation, sous une de mes impulsions, car je pensais que ma «~compétence~» dans le domaine apporterais un peu plus de dynamique au TPE. Malheureusement, la problématique était beaucoup trop portée sur l'aspect didactique de la programmation, et par conséquent, le TPE requérait des séances plénières pour avancer sérieusement. Séances que nous n'étions pas en mesure de fournir, car nous avons toujours eu du mal à nous retrouver régulièrement, pour diverses raisons. Le travail manquait de motivation et de dynamique. L'idée que je me faisais du TPE s'est effritée au contact de la réalité, ce que j'imaginais comme une recherche de la connaissance en groupe et une sorte de solidarité face à des problèmes intéressant était juste de la bureaucratie, chacun tentant de justifier son travail, moi y compris, et chacun ne faisant rien car la problématique était très peu sujette au travail en réalité.
  
  Heureusement Yann a su tenir tête à Madame Pivette pour changer la problématique en cours de route, car le TPE n'avait presque pas avancé. À partir de cette modification significative, qui sans changer le sujet, changeait grandement l'angle d'approche de celui-ci, nous avons pu créer une certaine dynamique de groupe, à partir d'une dynamique individuelle. En effet cette nouvelle problématique permettait plus aisément le travail individuel et la mise en commun ultérieure, ce qui a permis des progrès rapides et des séances plénières intéressantes.
  
  Durant ce travail j'ai eu l'occasion de confronter ma façon de penser avec d'autres personnes, qui elles ne programment pas, et qui de manière naïve, ont parfois apporté des solutions très efficaces. De plus ce fût pour moi l'occasion de programmer pour un projet véritable. On m'a fourni un cahier des charges, on vérifiait l'avancée, c'était une ambiance de production, bien qu'un peu détendue tout de même. La seule chose dommage est que l'on ai pas eu plus de temps pour effectuer des recherches et des expériences complémentaires afin d'améliorer la simulation. Il a été vraiment déplaisant de voir notre seule expérience en Biologie ne pas apporter de résultats utilisables. 
  
  Au final je trouve que ce TPE a été une assez bonne expérience, mais la confusion et le manque de but au départ a vraiment été pénible. Je pense qu'un sujet imposé et très restrictif aurait été plus productif car on aurait pu se baser sur un thème, quitte à dériver ensuite sur des sujets qui n'ont qu'un lien obscur. Par exemple donner des sujets «~bateau~» comme : «~Le stylo~», et ensuite partir dans la composition de l'encre aux différentes époques, les méthodes de création, et de faire tout un travail approfondi sur une chose commune et «~banale~». Mais peut-être est-ce plus compliqué à mettre en place et moins équitable que de permettre à tous de choisir son sujet. Néanmoins je ne me suis rendu compte de cela que vers le milieu du TPE, quand on ne pouvais plus changer le sujet, informer un peu plus les élèves serait sûrement appréciable\footnote{Aussi bien pour les élèves qui ont un sujet «~amusant~» que pour les correcteurs qui auront des sujets «~divertissants~» à corriger !}.
