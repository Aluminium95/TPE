Lors de la composition des équipes à la première séance, je n'avais pas vraiment choisi. J'ai pris celle où il restait de la place. C'est-à-dire avec Aliaume et Yann. Nous n'avions pas vraiment les mêmes goûts. Nous avons donc chacun exposés nos idées et nous sommes tombés sur plusieurs choix de sujets. Nous avons commencés avec une première problématique après une ou deux séances. Il était question de m'apprendre à programmer et de faire un programme. Il ne me serait jamais venu à l'idée d'apprendre à programmer sans ce TPE. Mais mes nombreuses absences et notre difficulté à trouver des horaires pour se regrouper perturber le travail. De plus je ne comprenais pas exactement ce que m'expliquais Aliaume quant au fonctionnement du logiciel de programmation Python. Nous abandonnâmes donc la première problématique pour une autre conseillé par les professeurs et par nous-mêmes. La deuxième, c'était seulement le programme et dire les problèmes que nous avions rencontrés. Seulement, il restait déjà peu de temps mais le programme avait bien avancé. Ils me donnèrent alors la responsabilité du design dont nous discutions à chaque entrevue ou en parlant par e-mail. Le programme était fini, il fallait s'attaquer à la rédaction du compte-rendu. Ce fut le plus dur car il a fallu accélérer le mouvement. Mais on y est quand même arrivé. En conclusion, je pourrais dire qu'avec la première problématique, j'exprimais peu d’intérêt. Mais à la deuxième en tant que designer, je me sentis un peu plus concerné. Je regrette que le travail fut plutôt laborieux de mon côté car je n'y connaissais rien du tout dans ce thème. Surtout quand Aliaume et Yan discutait du programme, et à certains moments, j'avais du mal à les suivre. Ce fut une bonne expérience qui m'a appris quelques petits trucs sur la programmation. J'ai pu aussi reprendre le cours sur les cellules où je n'avais pas était exemplaire pendant mes recherches et même en savoir plus. 
