De nos jours l’utilisation de l’informatique est omniprésente… Dans de nombreux domaines scientifiques elle est utilisée comme un outil  pratique qui offre une multitude d’opportunités. Les ordinateurs sont de plus en plus performants et offrent une puissance de calcul impressionnante qui peut maintenant aider dans de nombreux domaines. Nous avons donc inscrit la problématique du TPE dans cette logique de la résolution d’un problème scientifique, et plus particulièrement biologique, grâce à un programme de simulation. Au cours de ce rapport on suivra chronologiquement l’évolution du développement de ce programme : D’abord la première partie mettra en évidence les objectifs et les buts du logiciel. Elle sera consacrée à la vision globale des fonctionnalités du programme que l’on voulait implémenter, elle constitue en quelque sorte son cahier des charges. La deuxième partie est une description du développement du programme en lui-même, elle expliquera également les expériences et recherches auxquelles nous avons procédé pour obtenir les informations nécessaires à celui-ci. Enfin la troisième partie est en premier lieu la présentation du programme final mais également une explication des limites de la simulation et des problèmes qu’il peut éventuellement soulever. \\ \\ \\


\begin{quotation}
  \textit{
    Le travail sur ce programme fut une bonne expérience, tant sur le plan virtuel que les données recherchés pour le créer. Quelques problèmes dans la création de ce programme, mais c'était l'effet recherché d'après la problématique. Le gros de nos ambitions ont pu aboutir et chacun y a mis du sien.
  }
  \begin{flushright}
    \textsc{Sévigné Alfred}
  \end{flushright}
\end{quotation}


\begin{quotation}
  \textit{
    Ce TPE a été une véritable expérience pour nous et nous a ouvert les portes de la programmation qui restait jusqu'à présent un monde assez hermétique à nos yeux.
  }
  \begin{flushright}
    \textsc{Desmarais Yann}
  \end{flushright}
\end{quotation}

\begin{quotation}
  \textit{
    Ce TPE m'a donné l'occasion de travailler avec des véritables contraintes, et un véritable but.
    Il m'a aussi permis de travailler avec des non-programmeurs ce qui m'a ouvert l'esprit quand à la réalisation d'un projet en équipe.
  }
  \begin{flushright}
    \textsc{Lopez Aliaume}
  \end{flushright}
\end{quotation}
