\section{Objectifs}
Le but exact que nous nous sommes fixés pour le programme était l’étude du temps de vie et de division des cellules. Nous avons décidé pour plus de simplicité de nous en tenir à un seul type de cellule. Cela amena à des recherches qui nous orientèrent vers le choix des cellules de levures, qui sont des organismes eucaryotes dont le temps de division correspondait à notre échelle de temps. En effet les levures mettent deux heures pour se diviser une fois quand une cellule humaine en met vingt. Nos objectifs étaient qu’à l’aide du programme, à n’importe quel moment d’une culture de levure, on puisse savoir combien de cellules il y avait dans la colonie. Cela impliquait que le logiciel comprenne un compteur de temps et que l’on puisse stopper la simulation à tout moment pour voir l’état des cellules. A l’origine nous avions prévu deux environnements distincts dans l’interface ce qui permettait d’avoir une culture témoin et un autre test. Cette idée était motivée par le projet de soumettre aux levures des contraintes et de voir comment celle-ci y réagissait. Nous voulions en effet voir quelles étaient les différences entre une culture normale et une, par exemple, exposée à des agents mutagènes tels que les UV. Pour la représentation des cellules nous avions pensé à quelque chose de simple qui permettait une représentation sur un grand nombre de clones à l’échelle d’une colonie entière. Nous pension le menu comme une fenêtre classique avec une barre d’outil en haut ou apparaitraient les différentes fonctionnalités, les deux environnements de simulation et une barre dans le bas de la fenêtre qui afficherait les donnée de la simulation (nombres de cellules, temps de division etc). \\

\section{Recherches}

Pour réaliser cette vision première que nous avions du programme nous avons procédé il fallait réunir plusieurs conditions : 

\begin{itemize}
  \item Collecter les informations nécessaires à une simulation réaliste
  \item Faire nous même les expériences au cas où l’information n’est pas trouvée
  \item Trouver des outils informatiques performants et qui réponde à nos attentes 
  \item Trouver les fonctions impossibles à implémenter et en proposer d’autres en remplacement
\end{itemize}

Pour collecter les informations nécessaires nous avons pu utiliser internet (voir sitographie), quelque livres notamment le livre scolaire de SVT de première et de seconde (voir bibliographie), en se renseignant auprès de notre professeur de SVT voir de nos parents (les parents étaient pour chacun de nous soit biologistes soit informaticiens) pour certains détails.

Pour certaines informations concernant surtout les UV et leurs effets nous avons eu l’occasion de faire une expérience pour tester les colonies de levures exposés à ces rayonnements. Cette expérience et plus détaillée dans la partie deux dans le chapitre consacré aux UV.

Pour le choix des outils numérique nous nous sommes orientés vers un langage qui possède de nombreux avantage dont la simplicité et la rapidité. Malgré tout il avait des problèmes de portabilité. Cela nous empêcha donc de fournir le programme avec ce document, il sera présent lors de la présentation oral et sera pleinement testable…

Ensuite pour les fonctions qui n’ont pas aboutie, nous avons procédé par tâtonnement, en essayant de voir laquelle serait la plus cohérente avec l’objectif du programme et celles qui le rendrait plus pratique et facile d’accès. Nous avons par exemple ajouté le fait de pouvoir placer une grille de comptage dans chaque environnement pour mieux dénombrer les cellules, ou alors le fait de pouvoir zoomer pour voir le processus de division de plus près. 

\section{Problèmes}

Enfin il y a la question des problèmes pratiques qui interviennent dès le début du projet dans la mise en place de celui-ci. Ces problèmes étaient d’abord dans la simulation que nous voulions faire : être le plus réaliste possible en restant dans la mesure du réalisable. Trouver certaines données  a été assez difficile. Ensuite il y avait également la nécessité que le programme que nous voulions ne soit pas trop complexe à coder en définitive. En général ce ne fut pas une difficulté majeure, nous avons réussi à poser un projet réalisable et les rares fois où ce ne fut pas le cas nous nous sommes arrangés pour trouver des alternatives à ces fonctionnalités trop complexes. Enfin il y’avait les problèmes de choix personnels de chacun, nous avons dû nous mettre d’accord pour les orientations que nous souhaitions pour le logiciel. Au cours du développement et de la création du projet ce ne fut pas non plus très problématique du fait de la bonne entente du groupe et d’une communication assez régulière permettant la constatation des évolutions du programme quasiment en direct. En effet nous avions correspondances par mail importante qui se liait à l’utilisation d’une forge collaborative pour le code du programme : Github (voir sitographie).
