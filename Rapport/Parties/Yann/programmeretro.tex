\section{Rappel rapide du programme final}
Le programme se présentait sous la forme d’une fenêtre composé d’une barre d’outils, deux zones de simulations et d’une barre d’information. Les fonctions de bases étaient : Ajouter une cellule, retirer une cellule, mettre en pause, afficher la grille/éditer sa taille et enfin activer les Ultra-violet/les désactiver. Il y avait deux zones distinctes permettant une séparation utile pour faire un témoin et un test. La barre d’information donnait le nombre de cellules qui étaient présentes dans l’ensemble de l’environnement ou dans une zone délimité par la grille. Les Cellules étaient colorées différemment en fonction de leur âge. De plus il existait une fenêtre de paramétrage qui permettait de changer le temps de division, l’échelle de temps, l’échelle de taille et la répartition des cellules.

\section{Les différences avec notre vision du projet initial}
Le projet une fois mené à bien ressemblait en grande partie à ce que nous pension au début. Les quelques différences qui méritaient d’être notées sont : L’interface et l’aspect graphique la forme très simples de cellules, la gestion des mutations par UV, l’édition des caractéristique des cellules et certaines fonctions que nous n’avions pas prévu dès le début du TPE…

D’abord l’aspect graphique n’était pas exactement celui escompté, nous pensions pouvoir trouver le moyen de faire des boutons plus esthétique dans la barre d’outils. Cela restait un détail mais plus important dans ce registre était la représentation des cellules : Nous planchions au départ pour une représentation plus réaliste avec le noyau des cellules voir certains organites. Nous avons changé d’avis lorsque nous avons constaté que lorsque l’on observait les cellules dans leur globalité de loin ces inscriptions brouillaient la représentation. Ainsi nous avons décidé pour le programme un simple dégradé de couleur pour l’âge des cellules.

 Pour les mutations par le problème se trouve plutôt du côté des recherches peu fructueuses que nous avions menés pour trouver les taux. Nous n’avons trouvé que peu d’information et parfois contradictoires. Dans notre expérience l’un des objectif était de trouver ce taux de mutation, hors nous nous étions trompé de souche de levures : nous avions pris de la levure de boulanger qui ne forme pas de mutation visible à l’œil nu, nous n’avons constaté qu’après coup qu’il n’y avait que certaine souche comme les levures Ade2 qui change de couleur après mutation voir annexe levures.
 
Ensuite une des modifications qui ont été faites eut, cette fois ci, un effet bénéfique : Nos cellules ne possédaient que certaines caractéristiques fixes mais grâce à une modification nous avons créé un menu pour éditer celle-ci. Maintenant le programme a gagné une forme d’évolutivité car il peut théoriquement simuler différent types de cellules pour peux que l’utilisateur connaisse les caractéristiques de chacun d’eux. C’est une différence avec notre vision initiale puisque nous pensions que le programme se contenterait de paramètres uniques pour les cellules.

Enfin nous avons eu l’idée d’ajouter certaines fonctions qui rende le programme plus complexe. Il y a le fait de pouvoir choisir les échelles, le zoom, les grilles qui permettent un comptage facilité et les UV qui se contentent de détruire un certains nombres de cellules, ne possédant pas les pourcentages de mutations non-létale. Ces fonctionnalités n’étaient pas prévu dès le départ mais nous avons ressenti le besoin de les ajouter pour améliorer la qualité du programme et les opportunités qu’il ouvrait…
