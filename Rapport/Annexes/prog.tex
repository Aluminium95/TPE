Dans ce TPE dédié à la réalisation d'une simulation, nous allons devoir poser quelques bases de la programmation rapidement. Nous parlerons uniquement de la programmation en Vala, le langage choisi, de manière à se concentrer sur le fond, plutôt que sur la forme, car d'autres langages utilisent des concepts identiques mais avec des syntaxes parfois singulièrement différentes.

Nous tenterons de mettre en relation la programmation et les mathématiques, de manière à bien faire comprendre des concepts qui peuvent paraître abstrait de prime abord.

Cette annexe ne concerne aucunement l'algorithmique. L'algorithmique est un langage impératif, qui n'utilise que des nombres, et qui n'a aucun concept avancé. En somme c'est la base des langages, mais depuis il y a eu des évolutions, qui permettent de faire des choses beaucoup plus facilement, ou simplement d'aborder des problèmes différemment, nous utiliserons donc les fonctionnalités du Vala.

Si vous avez déjà quelques notions en algorithmie, vous n'aurez pas besoin de lire le chapitre sur les variables, mais le chapitre sur les fonctions vous sera utile quelque soit votre niveau.

\section{Variables}
  Les variables sont le concept de base d'un programme. Une variable est une boite, sur laquelle on pose une étiquette, et dans laquelle on met une valeur. 
  \begin{lstlisting}
    int a = 6;
  \end{lstlisting}
  Ce crée une boite appelée «~a~» qui est de type «~int~»\footnote{Integer : entier -> nombre. Pour une liste des types voir l'annexe \ref{DefTypes}}, et qui a pour valeur «~6~», le point virgule signifie la fin d'une commande. Cette opération est nommée affectation.
  
  En parlant d'opération, il en existe un certain nombre : 
  \begin{itemize}
    \item addition avec le symbole «~+~»
    \item soustraction avec le symbole «~-~»
    \item division avec le symbole «~/~»
    \item multiplication avec le symbole «~*~»
    \item modulo avec le symbole «~\%~»
  \end{itemize}
  
  La priorité opératoire suit les mêmes idiomes que les mathématiques. Les calculs peuvent comporter des variables : 
  \begin{lstlisting}
    int a = 3;
    int b = a * 6; // b = a * 6 => 3 * 6 => 18 
    a = b + 3;
  \end{lstlisting}
  
  Après avoir définit une variable, son contenu peut encore changer\footnote{Dans certains langages exotiques, ce n'est pas le cas}.

  Il existe différents types, mais le plus utilisé est le type nombre (int), et le type nombre à virgule (float).
  
\section{Contrôles}
  Encore une fois cette section peut être passée par les personnes qui ont déjà une idée sur l'algorithmie.
  Les structures de contrôle sont une partie d'un langage, voici des contrôles les plus courants en Vala :
  \begin{description}
    \item[ if (condition) / else if (condition) / else ] : si une condition est vraie, sinon si une autre est vraie, sinon si aucune des précédentes n'a été vérifiée
    \item[while (condition)] : tant qu'une condition est vérifier, exécuter le code suivant
    \item[for (a = depart; condition; a += ?)] : équivalent d'une somme algébrique mathématique, pour a qui vaut «~depart~», tant que la «~condition~» est vérifiée, et pour a qui à chaque tour de boucle augmente de ?
  \end{description}
  
  Exemples :
  \begin{lstlisting}
    int a = 5;
    while (a < 60) { // tant que a < 60
      a = a*2; //  on le multiplie par 2
    }
    
    if (a > 80) {
      a = 80;
    } else if ( a > 70) {
      a = 70;
    } else {
      a = 60;
    }
  \end{lstlisting}
  
  Le but de cette annexe n'est pas d'apprendre à programmer, mais à donner certaines connaissances indispensables à la compréhension du TPE. Du moment que le concept est saisi, vous pouvez passez à la suite.

\section{Fonctions}
  Les fonctions en programmation sont en réalité très similaires aux fonctions mathématiques. Si on définit par exemple en mathématique une fonction : 
  \[ f(x) = x^2 + 3x + 5 \]
  En Vala on la notera :
  \begin{lstlisting}
    // le int devant le f indique le type de retour
    int f (int x) {
      return x * x + 3 * x + 5;
    }
  \end{lstlisting}
  
  La grande différence est que en Vala il faut préciser le type d'entrée et le type de retour, car il n'y a pas que des entier quand on programme. Sinon tout est assez proche. Pour l'utilisation de notre fonction :
  \begin{lstlisting}
    int a = f(56); // a prend la valeur du retour de la fonction f
  \end{lstlisting}
  
  Mais une fonction en programmation est bien plus qu'une fonction mathématique, parce qu'elle peut faire autre chose que retourner une valeur, et avoir plus d'un paramètre, par exemple :
  \begin{lstlisting}
    int puissance (int x, int pow) {
      int i;
      int retour = 1;
      for (i = 0; i < pow; i++) {
        retour = retour * x;
      }
      return retour;
    }
  \end{lstlisting}
  
  Ici on utilise une boucle pour effectuer un certain nombre d'action, mais on peut aussi utiliser une fonction dite «~récursive~» c'est à dire qui s'appelle elle même : 
  \begin{lstlisting}
    int factorielle (int x) {
      if (x == 1) { 
        return 1;
      } else {
        return factorielle (x - 1) * x;
      }
    }
  \end{lstlisting}
  
  Attention, il faut bien définir au moins un cas ou la fonction ne s'appelle pas, sinon elle est infinie !
  
  Les fonctions peuvent donc prendre en paramètre n'importe quel type, retourner n'importe quel type, et faire un tas de calculs intermédiaires avant de retourner quelque chose, ou de ne rien retourner. Par exemple la fonction SDL qui affiche un rectangle sur l'écran ne retourne rien, elle se contente de faire un tas de calculs, de modifier des variables, et s'arrête quand le pixel est affiché.
