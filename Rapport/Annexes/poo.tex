
  \label{DefPOO}
  \section{Historique}
  	La définition de wikipédia est assez claire et concise, nous la complémenterons par un exemple : 
  	\begin{quotation}
  		La programmation orientée objet (POO), ou programmation par objet, est un paradigme de programmation informatique élaboré par Alan Kay dans les années 1970. Il consiste en la définition et l'interaction de briques logicielles appelées objets ; un objet représente un concept, une idée ou toute entité du monde physique, comme une voiture, une personne ou encore une page d'un livre. Il possède une structure interne et un comportement, et il sait communiquer avec ses pairs. Il s'agit donc de représenter ces objets et leurs relations ; la communication entre les objets via leurs relations permet de réaliser les fonctionnalités attendues, de résoudre le ou les problèmes.
  		\begin{flushright}
  			Wikipédia
  		\end{flushright}
  	\end{quotation}

    La programmation orientée objet introduit la notion de «~classe~». Une classe est 
    un patron, une définition d'un objet. Chaque «~classe~» définit la structure d'un type
    d'objet. Cette définition contient des variables «~membres~», et des fonctions membres dites «~méthodes~».
    
    Imaginons une Classe \texttt{humain}, qui a des variables \textit{internes} et
    des fonctions \textit{internes}. On pourra créer des objets à partir de ces patrons. C'est à dire 
    créer un être humain à partir de la classe qui définit \texttt{humain} pour conserver l'exemple. Chaque objet 
    à la même structure, mais peut évoluer différemment au cours de l’exécution. On peut créer
    deux humains avec une variable \texttt{couleur des yeux} par exemple. Pour le premier elle peut valoir \texttt{bleu} et pour le second \texttt{vert}. La différence entre une classe et une structure est qu'une Classe 
    peut contenir des variable ET des fonctions. Par exemple la fonction \texttt{direBonjour} de la 
    Classe \texttt{humain} utilise la variable \texttt{nom} et écrit : bonjour, je m'appelle + nom.
    Cette même fonction aura des effets différent sur différents objets.
    
    Une classe peut avoir comme variable membre n'importe quel type. Comme une classe est un type, une classe peut en contenir une autre, c'est même toute l'utilité en général.
  \section{Héritage}
    Une notion très importante qui vient avec la POO est l'héritage. Un exemple vaut tout les discours :
      \begin{itemize}
        \item Une classe \texttt{Humain}
        \item Une classe \texttt{Femme} qui hérite d'\texttt{Humain}
        \item Une classe \texttt{Homme} qui hérite d'\texttt{Humain}
      \end{itemize}
    
    La classe \texttt{Humain} définit des variables et fonctions propres aux humains. Ces fonctions sont réutilisées dans la classe \texttt{Homme} et la classe \texttt{Femme}, qui sont des humains, mais avec des caractéristiques supplémentaires et des réactions différentes. Au lieu de réécrire deux fois toutes les fonctions et attributs d'\texttt{Humain} dans les classes \texttt{Femme} et \texttt{Homme}, on créer une classe qui les regroupes, et on fait hériter \texttt{Homme} et \texttt{Femme} de celle ci.
    
    En plus de permettre une certaine paresse, cette méthode a le mérite de permettre une chose : \texttt{Homme} et \texttt{Femme} peuvent être considérés comme des \texttt{Humain}. Si on veut créer une fonction qui compare la taille de deux 
    personnes, on a juste à faire ceci :
    \begin{lstlisting}
      bool plus_grand_que (Humain p, Humain s) {
        if (p.taille > s.taille ) {
          return true;
        } else {
          return false;
        }
      }
    \end{lstlisting}
    
    On peut donner à cette fonction aussi bien deux \texttt{Femme}, deux \texttt{Homme} ou un \texttt{Homme} et un \texttt{Femme}. Étant donné qu'ils sont tous «~convertibles~» en \texttt{Humain}.
    
    Il existe d'autres principes dans la POO que nous n'aborderons pas ici simplement parce que c'est une annexe, qui même si elle est nécessaire à la compréhension, n'est pas le centre de ce TPE.
